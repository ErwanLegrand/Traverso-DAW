\section{Requirements}
This chapter describes how to arrange and write a Red Book compatible audio CD. Traverso uses \texttt{cdrdao} to actually write the CD, so this program must be installed on the system. \texttt{cdrdao} is available from the official repositories of all  major and up-to-date Linux distributions, it is thus recommended to install it via the distribution's package manager. It is also available for Microsoft Windows and Apple OS X (via fink and the MacPorts), but it requires some more effort to get installed on these systems. You can easily check if the tool is already installed by entering \texttt{cdrdao} on a console. If you get an output like this, you are ready to proceed:

\footnotesize
\begin{verbatim}
tux@linux:~$ cdrdao

Cdrdao version 1.2.2 - (C) Andreas Mueller <andreas@daneb.de>
  SCSI interface library - (C) Joerg Schilling
  Paranoia DAE library - (C) Monty

Check http://cdrdao.sourceforge.net/drives.html#dt for current 
driver tables.


Usage: cdrdao <command> [options] [toc-file]
command:
...
\end{verbatim}
\normalsize

If the command was not found, the following sections explain how to install \texttt{cdrdao} on various platforms.

\subsection{Linux}
Installing \texttt{cdrdao} on Linux is straight forward, since it is part of all major distributions. Use your distribution's package manager (e.\,g. Synaptic or Adept on (K)Ubuntu, Yast on SuSE), search for \texttt{cdrdao} and install the binary package. Alternatively, you can install it from a terminal. The commands will differ depending on the distribution. For (K,X)Ubuntu, enter the following lines in a terminal:

\begin{verbatim}
sudo apt-get update
sudo apt-get install cdrdao
\end{verbatim}

\subsection{Apple OS X}
Note: The installation on OS X requires a working internet connection, and the user must be allowed to get administrator privileges. You can install \texttt{cdrdao} from the MacPorts, but it can happen that only a source package is available for your version of OS X. First install MacPorts from \cite{macports}. Then open a terminal and enter:

\begin{verbatim}
sudo /opt/local/bin/port install cdrdao -v
\end{verbatim}
This will take a while, because several dependencies will be installed, too. But after the process has finished, you should be able to run \texttt{cdrdao} by entering:

\begin{verbatim}
/opt/local/bin/cdrdao
\end{verbatim}
which should give the same output as shown above. If something went wrong, please seek help from the MacPorts community.

\subsection{Microsoft Windows}
Traverso will install everything necessary to burn CDs automatically, including cdrdao. A working internet connection is required, because the latest cd-writing driver will be downloaded during the installation.

\section{Tracks and Markers}
There are basically two ways of defining tracks for a CD. Each song can be a track, or the entire CD can be arranged in the timeline of a song and tracks are defined by markers. Combinations of the two ways are also possible. Let's have a closer look at these two concepts.

\subsection{A Song is a CD-Track}
As you may have noticed, Traverso allows to have several songs in a project. Some people like this feature, as one can combine all songs of an album in one project, and still focus on one song at a time. If you want to write a CD containing all songs of your project, make sure you check the ``All Songs'' button in the export dialogue. Each song will be rendered to a track, from position 00:00:00 up to the end of the last audio clip, and consequently each song will become a track on the CD.

\subsection{CD in a timeline}
Sometimes it is important to fine-tune the transition from one track to the next, e.\,g. by adding a little bit of silence in between, or by fading the previous track into the next one. In that case it can be easier to arrange the entire CD in one timeline and split it into tracks using markers. Let's look at an example in order to show how this works. (Look at \FigT~\ref{fig_markers01} if you get lost with the explanations.) Open or create a project with only two audio clips. Suppose we want clip 1 to be track 1 on the CD, and clip 2 will be track 2. Position them on the first and maybe second track, starting at position 00:00:00, as you want to hear them on the CD. Leave some silence between the end of clip 1 and the beginning of clip 2. To get Traverso to start a new CD track there, position the mouse cursor on the gap between the two clips, and press \sact{M}. This adds a small triangle to the timeline at the position of the mouse cursor, and two more at position 00:00:00 and at the end of clip 2. The latter one is labelled ``End'', and it marks the end of the CD. You can shift it a bit further back if you don't want the CD to stop right there (remember you  can have reverb tails extending beyond the last audio sample, which you don't want to cut off).

\begin{figure}[t]
 \centering\includegraphics[width=\textwidth]{images/markers01}
 \caption{If a CD is arranged in one song, markers can be used to define CD tracks. Always keep the ones at position 00:00:00 and at the end.}
 \label{fig_markers01}
\end{figure}

These triangles are CD track markers, and they can be moved, added, and deleted freely (press \sact{Q} on the timeline to list all available functions.) However, it is also possible to create setups which don't make sense. E.\,g. only having one track marker in the timeline. In such cases, Traverso tries to guess the most sensible solution, and adds markers on the fly at positions it considers appropriate (which is usually at position 00:00:00 and after the last sample of the song containing audio data). Traverso also supports CD-text, which can be entered in the marker dialogue ``Views $\rightarrow$ Marker Dialog'' (\FigB~\ref{fig_marker-editor}). It is also possible to export the table of contents of the CD as an HTML file from this dialog. Album-wide CD-text can be entered in the project settings, opened from ``Project $\rightarrow$ Project Manager'', in the tab ``CD Text''.

\begin{figure}[ht]
 \centering\includegraphics[width=0.45\textwidth]{images/marker-editor}
 \caption{The marker dialogue opened from ``Views $\rightarrow$ Marker Dialog'' allows to add CD-text, modify the markers, and export the table of contents as an HTML file.}
 \label{fig_marker-editor}
\end{figure}

Once the CD is laid out to your satisfaction, press \dact{RETURN} on the song to open the export dialogue (\FigB~\ref{fig_exportdlg}). You can either choose to export the project as audio files to the harddisk, or burn a CD. If you want to export to harddisk, you can choose between several audio formats. The most common one is WAVE, and depending on your plans with the exported files, you can use different sample formats (bit depths). 16 bit is ideal if you want to burn the files on CD later on. If you want to go on processing the files, use 32 bit float format instead. If you want to archive the files, use the FLAC codec, which is a lossless compression format.

If you want to burn a CD, you must also decide if you want to burn the current song (using markers to define CD tracks), or the entire project (each song becomes a track). If you check ``Export to disk only'', no CD will be written, but only a *.toc file and *.wav files for \texttt{cdrdao}.

Note for OS X users: CD writing support is still experimental. You can choose between several burning devices: IODVDServices, IODVDServices/2, IOCompactDiscServices, IOCompactDiscServices/2. These are hard-coded, so you probably don't have all of them installed. IOCompactDiscServices should only be used for old drives without DVD reading support. If you have multiple DVD drives, use IODVDServices or IODVDServices/2 to access the first and the second drive. In most cases IODVDServices will be the only working solution.

\begin{figure}[t]
 \centering\includegraphics[width=\textwidth]{images/exportdlg}
 \caption{\dact{RETURN} opens a dialogue to export the project (either the current song or the entire project) to audio files, or burn a CD.}
 \label{fig_exportdlg}
\end{figure}

