\begin{itemize}
 \item \textit{Playback is not smooth and I get lots of xruns}\\
  If you are using an Intel onboard sound chip, you may have to set the Number of Periods to 3 instead of 2. This can be done in the ``Audio Driver'' page of the Preferences dialogue, or in your jackd configuration (e.\,g. qjackctl).

 \item \textit{I can't hear anything}\\
  If playback seems to be working (VUMeters indicate a signal), but you hear no sound, check if the ``Null Driver'' is loaded. If yes, load a different driver, e.\,g. ALSA on Linux, or PortAudio on OS X or Windows, and try again. If you are using the jackd driver, make sure the output of Traverso is connected to a hardware output. This is described in chapter \ref{sect_setup}.

 \item \textit{I can't load the driver}\\
  One problem could be that the buffer size specified in the driver settings is too large for your sound card. Try reducing the latency a bit. It can also happen that the sound system is blocked by another application or demon. If you are using KDE, make sure aRTs terminates automatically after a few seconds, and try to shut down any other application blocking the sound card.

 \item \textit{My mouse stops moving when I press a key}\\
  In order to avoid accidental input, some laptops disable the trackpad during key presses. This of course spoils the concept of soft selections completely, but fortunately it can be turned off in most cases. If you are using Linux, the \texttt{mouseemu} demon handles the feature. You can either turn off \texttt{mouseemu} by running the command 
  \begin{verbatim}
  sudo /etc/init.d/mouseemu stop
  \end{verbatim}
  and check if the problem is solved, or you can edit the configuration file \texttt{/etc/defaults/mouseemu} and change the value 300 in the line \texttt{TYPING\_BLOCK="-typing-block 300"} to 0. If there is a comment sign ``\#'' at the beginning of the line, remove it. Then restart \texttt{mouseemu} by typing
  \begin{verbatim}
  sudo /etc/init.d/mouseemu restart
  \end{verbatim}

 If you are using OS X, disable the feature ``ignore accidental trackpad input'' in the system settings.

 \item \textit{Traverso can't set the audio thread to real time priority}\\
 Some Linux distributions don't allow to set the priority of the audio thread to real time. However, a too low priority can lead to xruns and dropouts during recording and playback. To avoid this, add the following lines to the file \texttt{/etc/security/limits.conf}, or if the lines are already there, change the numbers accordingly:

\begin{tabular}{llll}
\texttt{@audio} & \texttt{-} & \texttt{rtprio} & \texttt{90}\\
\texttt{@audio} & \texttt{-} & \texttt{nice} & \texttt{-10}\\
\texttt{@audio} & \texttt{-} & \texttt{memlock} & \texttt{3000000}\\
\end{tabular}

Then restart your computer.

\end{itemize}


